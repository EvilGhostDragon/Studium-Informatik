
\documentclass[12pt]{article}
\usepackage[ngerman]{babel}
\begin{document}
\begin{titlepage}
\begin{center}
Mitschrift WS 19/20 - Informatik
\end{center}
\tableofcontents

\end{titlepage}
\section{Rechenarchitektur}
\subsubsection{PS: Rechenarchitektur}



\newpage
\section{Funktionale Programmierung}
\subsection{VO - Mitschrift}


\today\\\\
Haskell associaiert verschiedene Operationen anders: left to right /  right to left
Heskell: Immer if \textbf{und} else und nie ohne else\\
device-and-conquer: Große Instanz in kleinere zerlegen und diese zuvor rechnen und dannach die größere


\newpage
\section{Einführug in die theoretische Informatik}
\subsection{SL - Aufgaben}
\subsubsection{Aufgabe 1 - \today}
\begin{enumerate}
\item Aufgabe \\\\
Vereinigung: Elemente beider Mengen:$ A \cup B = $ \\
Durchschnitt: Elemente welche in $M_1$ und $M_2$ vorkommt\\
Differenz: Elemente der Menge $M_1$ abzüglich der Elemente voon $M_2$\\


\end{enumerate}
\newpage
\section{Einführug in die Programmierung}


\newpage
\section{Lineare Algebra}
\subsection{Gauße Algorithmus}
Homogen: b sind 0
Koeffizientenmatrix: muss nich Quatratisch sein -> hängt von der Lösungsmenge ab.\\
erweiterte Koeffizientenmatrix: mit b am Ende. Kann in der Praxis nicht von der Koeffizientenmatrix unterschieden werden\\\\
Lösungsmenge: Def: $L(A,b) = {(c_1,..,c_2) \epsilon\Re$}


\end{document}